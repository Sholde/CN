\documentclass[12pt, letterpaper]{article}
\usepackage[utf8]{inputenc}

\title{Réponse Exercice 1}
\author{Nicolas Bouton}

\begin{document}

\begin{titlepage}
\maketitle
\end{titlepage}

\section{Exercice 1}

\begin{itemize}
\item $X = [4,\; 5, \; 6] \rightarrow$ vecteur ligne
\item $X = [4 \; 5 \; 6] \rightarrow$ vecteur ligne
\item $X\;X = [4;5;6] \rightarrow$ erreur de syntax
\item $X = [4;5;6] \rightarrow$ vecteur colonne
\item $X = [4, \; 5, \; 6] \rightarrow$ vecteur colonne
\item $X = [4 \; 5 \; 6] \; ; \rightarrow$ rien
\item $X = [4:7] \rightarrow$ créer un vecteur ligne de $7-4$ valeurs avec des valeurs de 4 à 6
\item $Y = [sqrt(3) \quad 12+3\%i \quad \%pi \quad -1] \rightarrow$ créer un vecteur ligne avec le résultat de la fonction carré, un nombre complexe, la valeur pi et un nombre négatif
\item $X*Y \rightarrow$ produit scalaire impossible, le nombre de colonne de X $\neq$ le nombre de ligne de Y
\item $X'*Y \rightarrow$ produit scalaire de la transposé de X avec Y
\item $X.*Y \rightarrow$ transforme X en matrixe diagonale ?, et ensuite fait le produit scalaire avec Y
\item $C = [5 \; : \; 7; \; 1, \; 2 \; 3; \; 7 \; : \; 9] \rightarrow$ le : permet de faire une suite de nombre incrémenté de 1, et le ; permet de créer une nouvelle ligne, donc C est de taillt $3x3$
\item $B(3, \; 2) = 13 \rightarrow$ set la valeur 13 à la case qui correspond à la 3ème ligne et la 2ème colonne
\item $size(B) \rightarrow$ donne la taille de la matrice, ici $3x2$
\item $C = zeros(3, \; 2) \rightarrow$ créer une matrice de taille $3x2$ et initialise la avec des 0
\item $X = [4:0.5:7] \rightarrow$ on choisit un pas de $0,5$
\item $Z = X.*X-10*X; \; plot(X, \; Z, \; 'o'); \rightarrow$ ; permet de séparer 2 instruction, on peut dessiner des graphes avec plot
\item $A' \rightarrow$ transposé
\item $det(A)$ calcul le déterminant
\item $inv(A)$ calcul l'inverse
\item $A*A*A \rightarrow$ calcul $A^3$
\item $A\\b \rightarrow$ je ne sais pas, peut le vecteur B tel que $AX = B$
\end{itemize}

\end{document}

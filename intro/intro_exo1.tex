\documentclass[12pt, letterpaper]{article}
\usepackage[utf8]{inputenc}

\title{Réponse Exercice 1}
\author{Nicolas Bouton}

\begin{document}

\begin{titlepage}
\maketitle
\end{titlepage}

\section{Exercice 1}

\begin{itemize}
\item $X = [4,\; 5, \; 6] \rightarrow$ vecteur ligne
\item $X = [4 \; 5 \; 6] \rightarrow$ vecteur ligne
\item $X\;X = [4;5;6] \rightarrow$ erreur de syntax
\item $X = [4;5;6] \rightarrow$ vecteur colonne
\item $X = [4, \; 5, \; 6] \rightarrow$ vecteur colonne
\item $X = [4 \; 5 \; 6] \; ; \rightarrow$ rien
\item $X = [4:7] \rightarrow$ créer un vecteur ligne de $7-4$ valeurs avec des valeurs de 4 à 6
\item $Y = [sqrt(3) \quad 12+3\%i \quad \%pi \quad -1] \rightarrow$ créer un vecteur ligne avec le résultat de la fonction carré, un nombre complexe, la valeur pi et un nombre négatif
\item $X*Y \rightarrow$ produit scalaire impossible, le nombre de colonne de X $\neq$ le nombre de ligne de Y
\item $X'*Y \rightarrow$ produit scalaire de la transposé de X avec Y
\item $X.*Y \rightarrow$ transforme X en matrixe diagonale ?, et ensuite fait le produit scalaire avec Y
  
\end{itemize}

\end{document}

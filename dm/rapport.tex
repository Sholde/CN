\documentclass{article}

% Package
\usepackage{amsmath}

% Title information
\title{DM Calcul Numérique}
\author{BOUTON Nicolas}

% Begin document 
\begin{document}

\maketitle

% Exercice 1
\section*{Exercice 1}

\begin{enumerate}

  % Question 1
\item 

  Utilisons l'interpolation lagragienne.\newline
  On a :
  $$ f(t) \approx p(t) $$

  On veut maintenant approcher l'intégrale de f, on va donc utilisé
  l'intégrale du polynôme de lagrange :

  \begin{equation*}
    \int_a^b {f(t) dt} \approx \int_a^b {p(t) dt}
  \end{equation*}

  Et on a :

  \begin{equation*}
    \int_a^b {p(t) dt} = \sum_{i = 1}^{n} {f(t_i)w_i}
  \end{equation*}

  Avec \textbf{n} le nombre de point et $w_i$ :

  \begin{equation*}
    w_i = \int_a^b {L_i(t)dt}
  \end{equation*}

Essayons donc de calculer les polynôme de Lagrange associés aux points
: \newline
0, $\frac{1}{3}$, $\frac{2}{3}$, 1.

% L0 avec le point 0
\underline{Point $x = 0$ :}
\begin{equation*}
  \begin{split}
    L_0 = & \frac{(x - \frac{1}{3}) (x - \frac{2}{3}) (x - 1)}
    {(0 - \frac{1}{3}) (0 - \frac{2}{3}) (0 - 1)} \\
    = & \frac{(x^2 - \frac{2}{3}x - \frac{1}{3}x + \frac{2}{6}) (x - 1)}
    {(- \frac{2}{9})} \\
    = & \frac{(x^3 - x^2 + \frac{2}{6}x) - (x^2 - x + \frac{2}{6}) }
    {(- \frac{2}{9})} \\
    = & \frac{(x^3 - 2x^2 + \frac{8}{6}x - \frac{2}{6}) }
    {(- \frac{2}{9})} \\
    = & - \frac{ 9 (x^3 - 2x^2 + \frac{8}{6}x - \frac{2}{6}) }
    {2} \\
  \end{split}
\end{equation*}

% L1 avec le point 1/3
\underline{Point $x = \frac{1}{3}$ :}
\begin{equation*}
  \begin{split}
    L_1 = & \frac{(x - 0) (x - \frac{2}{3}) (x - 1)}
    {(\frac{1}{3} - 0 ) (\frac{1}{3} - \frac{2}{3}) (\frac{1}{3} - 1)}
    \\
    = & \frac{(x^2 - \frac{2}{3}x) (x - 1)}
    {(\frac{1}{3}) (- \frac{1}{3}) (- \frac{2}{3})} \\
    = & \frac{(x^3 - \frac{2}{3}x^2) - (x^2 - \frac{2}{3}x)}
    {(\frac{2}{27})} \\
    = & \frac{(x^3 - \frac{5}{3}x^2 + \frac{2}{3}x)}
    {(\frac{2}{27})} \\
    = & \frac{27 (x^3 - \frac{5}{3}x^2 + \frac{2}{3}x)}
    {2} \\
  \end{split}
\end{equation*}


% L2 avec le point 2/3
\underline{Point $x = \frac{2}{3}$ :}
\begin{equation*}
  \begin{split}
    L_2 = & \frac{(x - 0) (x - \frac{1}{3}) (x - 1)}
    {(\frac{2}{3} - 0 ) (\frac{2}{3} - \frac{1}{3}) (\frac{2}{3} - 1)}
    \\
    = & \frac{(x^2 - \frac{1}{3}x) (x - 1)}
    {(\frac{2}{3}) (\frac{1}{3}) (- \frac{1}{3})}
    \\
    = & \frac{(x^3 - \frac{1}{3}x^2) - (x^2 - \frac{1}{3}x)}
    {- \frac{2}{27}}
    \\
    = & \frac{(x^3 - \frac{4}{3}x^2 + \frac{1}{3}x)}
    {- \frac{2}{27}}
    \\
    = & - \frac{27 (x^3 - \frac{4}{3}x^2 + \frac{1}{3}x)}
    {2}
    \\
  \end{split}
\end{equation*}

% L3 avec le point 1
\underline{Point $x = 1$ :}
\begin{equation*}
  \begin{split}
    L_3 = & \frac{(x - 0) (x - \frac{1}{3}) (x - \frac{2}{3})}
    {(1 - 0 ) (1 - \frac{1}{3}) (1 - \frac{2}{3})}
    \\
    = & \frac{(x^2 - \frac{1}{3} x) (x - \frac{2}{3})}
    {(\frac{2}{3}) (\frac{1}{3})}
    \\
    = & \frac{(x^3 - \frac{1}{3} x^2) - (\frac{2}{3}x^2 - \frac{2}{3}\frac{1}{3} x)}
    {(\frac{2}{9})}
    \\
    = & \frac{(x^3 - x^2 + \frac{2}{9} x)}
    {(\frac{2}{9})}
    \\
    = & \frac{ 9 (x^3 - x^2 + \frac{2}{9} x)}
    {2}
    \\
  \end{split}
\end{equation*}

Calculons maintenant leurs intégrale :

% I(L0) avec le point 0
\underline{Point $x = 0$ :}
\begin{equation*}
  \begin{split}
    \int_0^1 {L_0(t)dt} = & \int_0^1 {- \frac{ 9 (t^3 - 2t^2 +
        \frac{8}{6}t - \frac{2}{6}) } {2} dt} \\
    = & - \frac{9}{2} \int_0^1 {\left(t^3 - 2t^2 +
      \frac{8}{6}t - \frac{2}{6} \right)  dt} \\
    = & - \frac{9}{2} \left[ \frac{1}{4}t^4 - \frac{2}{3}t^3 +
      \frac{8}{12}t^2 - \frac{2}{6}t  \right]_0^1 \\
    = & - \frac{9}{2} \left(\frac{1}{4} * 1^4 - \frac{2}{3} * 1^3 +
    \frac{8}{12} * 1^2 - \frac{2}{6} * 1 \right)  \\
    = & - \frac{9}{2} \left(\frac{1}{4} - \frac{2}{3} +
    \frac{8}{12} - \frac{2}{6} \right)  \\
    = & - \frac{9}{2} \left(\frac{3}{12} - \frac{8}{12} +
    \frac{8}{12} - \frac{4}{12} \right)  \\
    = & - \frac{9}{2} \left(- \frac{1}{12} \right)  \\
    = & \frac{9}{24} \\
    = & \frac{3}{8}
  \end{split}
\end{equation*}

% I(L1) avec le point 1/3
\underline{Point $x = \frac{1}{3}$ :}
\begin{equation*}
  \begin{split}
    \int_0^1 {L_1(t) dt} = & \int_0^1 {\frac{27 (t^3 - \frac{5}{3}t^2 + \frac{2}{3}t)}
      {2} dt} \\
    = & \frac{27}{2} \int_0^1 { \left( t^3 - \frac{5}{3}t^2 + \frac{2}{3}t \right) 
      dt} \\
    = & \frac{27}{2} \left[ \left( \frac{1}{4}t^4 - \frac{5}{9}t^3 +
      \frac{2}{6}t^2 \right) \right]_0^1 \\
    = & \frac{27}{2} \left( \frac{1}{4} - \frac{5}{9} + \frac{2}{6} \right)  \\
    = & \frac{27}{2} \left( \frac{9}{36} - \frac{20}{36} + \frac{12}{36} \right)  \\
    = & \frac{27}{2} \left(  \frac{1}{36} \right)  \\
    = & \frac{27}{72} \\
    = & \frac{3}{7}
  \end{split}
\end{equation*}

% I(L2) avec le point 2/3
\underline{Point $x = \frac{2}{3}$ :}
\begin{equation*}
  \begin{split}
    \int_0^1 {L_2(t) dt} = & \int_0^1 {- \frac{27 (t^3 - \frac{4}{3}t^2 + \frac{1}{3}t)}
    {2} dt }
    \\
    = & - \frac{27}{2} \int_0^1 { \left( t^3 - \frac{4}{3}t^2 +
      \frac{1}{3}t \right) dt }
    \\
    = & - \frac{27}{2} \left[ \frac{1}{4}t^4 - \frac{4}{9}t^3 +
      \frac{1}{6}t^2 \right]_0^1
    \\
    = & - \frac{27}{2} \left( \frac{1}{4} - \frac{4}{9} + \frac{1}{6} \right) 
    \\
    = & - \frac{27}{2} \left( \frac{9}{36} - \frac{16}{36} +
    \frac{6}{36} \right)
    \\
    = & - \frac{27}{2} \left( - \frac{1}{36} \right) 
    \\
    = & \frac{27}{72}
    \\
    = & \frac{3}{7}
    \\
  \end{split}
\end{equation*}

% I(L3) avec le point 1
\underline{Point $x = 1$ :}
\begin{equation*}
  \begin{split}
    \int_0^1 {L_3(t) dt} = & \int_0^1 {\frac{ 9 (t^3 - t^2 + \frac{2}{9} t)}
    {2} dt}
    \\
    = & \frac{9}{2} \int_0^1 { \left(t^3 - t^2 + \frac{2}{9} t \right) dt}
    \\
    = & \frac{9}{2} \left[ \frac{1}{4}t^4 - \frac{1}{3}t^3 +
      \frac{2}{18} t^2 \right]_0^1
    \\
    = & \frac{9}{2} \left(\frac{1}{4} - \frac{1}{3} +
      \frac{2}{18} \right)
    \\
    = & \frac{9}{2} \left( \frac{9}{36} - \frac{12}{36} +
      \frac{4}{36} \right)
    \\
    = & \frac{9}{2} \left( \frac{1}{36} \right)
    \\
    = & \frac{1}{7}
    \\
  \end{split}
\end{equation*}

\underline{Résultat :}

\begin{equation*}
    w_0 = \frac{3}{8},
    w_1 = \frac{3}{7},
    w_2 = \frac{3}{7},
    w_3 = \frac{1}{7}
\end{equation*}


\end{enumerate}

% Exercice 2
\section*{Exercice 2}

\begin{enumerate}

  % Question 1
\item

  \begin{enumerate}

    % Question a
  \item

    D'après la méthode de Gauss-Legendre on a :
    \begin{equation*}
      w_i = \frac{2}{(1 - x_i^2) P'_n(x_i)^2}
    \end{equation*}

    Ici on a $n = 3$, donc on a la formule suivante :

    \begin{equation*}
      P_3(x) = \frac{1}{2} (5x^3 - 3x)
    \end{equation*}

    Et sa dérivé est donc :

    \begin{equation*}
      P'_3(x) = \frac{1}{2} (15x^2 - 3)
    \end{equation*}

    \underline{Nous pouvons désormais calculé les $wi$ :} \newline

    $i = 0, x_0 = 0 :$
    
    \begin{equation*}
      \begin{split}
        w_0 = & \frac{2}{(1 - 0^2) \left( \frac{1}{2} (15* 0^2 -
          3) \right)^2} \\
        = & \frac{2}{\left( \frac{1}{2} (- 3) \right)^2} \\
        = & \frac{2}{\frac{9}{4}} \\
        = & \frac{8}{9}
      \end{split}
    \end{equation*}

    $i = 1, x_1 = \frac{1}{3} :$

    \begin{equation*}
      \begin{split}
        w_1 = & \frac{2}{(1 - \frac{1}{3}^2) \left( \frac{1}{2} (15*
          \frac{1}{3}^2 - 3) \right)^2} \\
        = & \frac{2}{(1 - \frac{1}{9}) \left( \frac{1}{2} (15*
          \frac{1}{9} - 3) \right)^2} \\
        = & \frac{2}{(\frac{8}{9}) \left( \frac{1}{2} (\frac{5}{3} -
          3) \right)^2} \\
        = & \frac{2}{(\frac{8}{9}) \left( \frac{1}{2} (- \frac{4}{3})
          \right)^2} \\
        = & \frac{2}{(\frac{8}{9}) \left(- \frac{4}{6} \right)^2} \\
        = & \frac{2}{(\frac{8}{9}) \left(- \frac{2}{3} \right)^2} \\
        = & \frac{2}{(\frac{8}{9}) \left(\frac{4}{9} \right)} \\
        = & \frac{2}{(\frac{32}{81}) } \\
        = & \frac{162}{32} \\
        = & \frac{81}{16}
      \end{split}
    \end{equation*}

    $i = 2, x_2 = \frac{2}{3} :$

    \begin{equation*}
      \begin{split}
        w_2 = & \frac{2}{(1 - \frac{2}{3}^2) \left( \frac{1}{2} (15*
          \frac{2}{3}^2 - 3) \right)^2} \\
        = & \frac{2}{(1 - \frac{4}{9}) \left( \frac{1}{2} (15*
          \frac{4}{9} - 3) \right)^2} \\
        = & \frac{2}{(\frac{5}{9}) \left( \frac{1}{2} (\frac{20}{3} -
          3) \right)^2} \\
        = & \frac{2}{(\frac{5}{9}) \left( \frac{1}{2} (- \frac{10}{3})
          \right)^2} \\
        = & \frac{2}{(\frac{5}{9}) \left(- \frac{10}{6} \right)^2} \\
        = & \frac{2}{(\frac{5}{9}) \left(- \frac{5}{3} \right)^2} \\
        = & \frac{2}{(\frac{5}{9}) \left(\frac{25}{9} \right)} \\
        = & \frac{2}{(\frac{125}{81}) } \\
        = & \frac{162}{125}
      \end{split}
    \end{equation*}

    $i = 3, x_3 = 1 :$

    \begin{equation*}
      \begin{split}
        w_3 = & \frac{2}{(1 - 1^2) \left( \frac{1}{2} (15* 1^2 -
          3) \right)^2} \\
        = & 0
      \end{split}
    \end{equation*}


  \end{enumerate}

  
\end{enumerate}

\end{document}

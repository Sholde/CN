\documentclass{article}

% Package
\usepackage{amsmath}

% Title information
\title{DM Calcul Numérique}
\author{BOUTON Nicolas}

% Begin document 
\begin{document}

\maketitle

% Exercice 1
\section*{Exercice 1}

\begin{enumerate}

  % Question 1
\item 

  Utilisons l'interpolation lagragienne.\newline
  On a :
  $$ f(t) \approx p(t) $$

  On veut maintenant approcher l'intégrale de f, on va donc utilisé
  l'intégrale du polynôme de lagrange :

  \begin{equation*}
    \int_a^b {f(t) dt} \approx \int_a^b {p(t) dt}
  \end{equation*}

  Et on a :

  \begin{equation*}
    \int_a^b {p(t) dt} = \sum_{i = 1}^{n} {f(t_i)w_i}
  \end{equation*}

  Avec \textbf{n} le nombre de point et $w_i$ :

  \begin{equation*}
    w_i = \int_a^b {L_i(t)dt}
  \end{equation*}

Essayons donc de calculer les polynôme de Lagrange associés aux points
: \newline
0, $\frac{1}{3}$, $\frac{2}{3}$, 1.

% L0 avec le point 0
\underline{Point $x = 0$ :}
\begin{equation*}
  \begin{split}
    L_0 = & \frac{(x - \frac{1}{3}) (x - \frac{2}{3}) (x - 1)}
    {(0 - \frac{1}{3}) (0 - \frac{2}{3}) (0 - 1)} \\
    = & \frac{(x^2 - \frac{2}{3}x - \frac{1}{3}x + \frac{2}{9}) (x - 1)}
    {(- \frac{2}{9})} \\
    = & \frac{(x^3 - x^2 + \frac{2}{9}x) - (x^2 - x + \frac{2}{9}) }
    {(- \frac{2}{9})} \\
    = & \frac{(x^3 - 2x^2 + \frac{11}{9}x - \frac{2}{9}) }
    {(- \frac{2}{9})} \\
    = & - \frac{ 9 (x^3 - 2x^2 + \frac{11}{9}x - \frac{2}{9}) }
    {2} \\
    = & - \frac{9}{2}x^3 + 9 x^2 - \frac{11}{2}x + 1
  \end{split}
\end{equation*}

% L1 avec le point 1/3
\underline{Point $x = \frac{1}{3}$ :}
\begin{equation*}
  \begin{split}
    L_1 = & \frac{(x - 0) (x - \frac{2}{3}) (x - 1)}
    {(\frac{1}{3} - 0 ) (\frac{1}{3} - \frac{2}{3}) (\frac{1}{3} - 1)}
    \\
    = & \frac{(x^2 - \frac{2}{3}x) (x - 1)}
    {(\frac{1}{3}) (- \frac{1}{3}) (- \frac{2}{3})} \\
    = & \frac{(x^3 - \frac{2}{3}x^2) - (x^2 - \frac{2}{3}x)}
    {(\frac{2}{27})} \\
    = & \frac{(x^3 - \frac{5}{3}x^2 + \frac{2}{3}x)}
    {(\frac{2}{27})} \\
    = & \frac{27 (x^3 - \frac{5}{3}x^2 + \frac{2}{3}x)}
    {2} \\
  \end{split}
\end{equation*}


% L2 avec le point 2/3
\underline{Point $x = \frac{2}{3}$ :}
\begin{equation*}
  \begin{split}
    L_2 = & \frac{(x - 0) (x - \frac{1}{3}) (x - 1)}
    {(\frac{2}{3} - 0 ) (\frac{2}{3} - \frac{1}{3}) (\frac{2}{3} - 1)}
    \\
    = & \frac{(x^2 - \frac{1}{3}x) (x - 1)}
    {(\frac{2}{3}) (\frac{1}{3}) (- \frac{1}{3})}
    \\
    = & \frac{(x^3 - \frac{1}{3}x^2) - (x^2 - \frac{1}{3}x)}
    {- \frac{2}{27}}
    \\
    = & \frac{(x^3 - \frac{4}{3}x^2 + \frac{1}{3}x)}
    {- \frac{2}{27}}
    \\
    = & - \frac{27 (x^3 - \frac{4}{3}x^2 + \frac{1}{3}x)}
    {2}
    \\
  \end{split}
\end{equation*}

% L3 avec le point 1
\underline{Point $x = 1$ :}
\begin{equation*}
  \begin{split}
    L_3 = & \frac{(x - 0) (x - \frac{1}{3}) (x - \frac{2}{3})}
    {(1 - 0 ) (1 - \frac{1}{3}) (1 - \frac{2}{3})}
    \\
    = & \frac{(x^2 - \frac{1}{3} x) (x - \frac{2}{3})}
    {(\frac{2}{3}) (\frac{1}{3})}
    \\
    = & \frac{(x^3 - \frac{1}{3} x^2) - (\frac{2}{3}x^2 - \frac{2}{3}\frac{1}{3} x)}
    {(\frac{2}{9})}
    \\
    = & \frac{(x^3 - x^2 + \frac{2}{9} x)}
    {(\frac{2}{9})}
    \\
    = & \frac{ 9 (x^3 - x^2 + \frac{2}{9} x)}
    {2}
    \\
  \end{split}
\end{equation*}

Calculons maintenant leurs intégrale :

% I(L0) avec le point 0
\underline{Point $x = 0$ :}
\begin{equation*}
  \begin{split}
    \int_0^1 {L_0(t)dt} = & \int_0^1 {\left( - \frac{9}{2}t^3 + 9 t^2
      - \frac{11}{2}t + 1 \right) dt}\\
    = &  \left[ - \frac{9}{8}t^4 + \frac{9}{3} t^3
      - \frac{11}{4} t^2 + t  \right]_0^1 \\
    = & - \frac{9}{8} * 1^4 + \frac{9}{3} * 1^3 -
    \frac{11}{4} * 1^2 + 1 \\
    = & \frac{3}{24} \\
    = & \frac{1}{8}
  \end{split}
\end{equation*}

% I(L1) avec le point 1/3
\underline{Point $x = \frac{1}{3}$ :}
\begin{equation*}
  \begin{split}
    \int_0^1 {L_1(t) dt} = & \int_0^1 {\frac{27 (t^3 - \frac{5}{3}t^2 + \frac{2}{3}t)}
      {2} dt} \\
    = & \frac{27}{2} \int_0^1 { \left( t^3 - \frac{5}{3}t^2 + \frac{2}{3}t \right) 
      dt} \\
    = & \frac{27}{2} \left[ \left( \frac{1}{4}t^4 - \frac{5}{9}t^3 +
      \frac{2}{6}t^2 \right) \right]_0^1 \\
    = & \frac{27}{2} \left( \frac{1}{4} - \frac{5}{9} + \frac{2}{6} \right)  \\
    = & \frac{27}{2} \left( \frac{9}{36} - \frac{20}{36} + \frac{12}{36} \right)  \\
    = & \frac{27}{2} \left(  \frac{1}{36} \right)  \\
    = & \frac{27}{72} \\
    = & \frac{3}{8}
  \end{split}
\end{equation*}

% I(L2) avec le point 2/3
\underline{Point $x = \frac{2}{3}$ :}
\begin{equation*}
  \begin{split}
    \int_0^1 {L_2(t) dt} = & \int_0^1 {- \frac{27 (t^3 - \frac{4}{3}t^2 + \frac{1}{3}t)}
    {2} dt }
    \\
    = & - \frac{27}{2} \int_0^1 { \left( t^3 - \frac{4}{3}t^2 +
      \frac{1}{3}t \right) dt }
    \\
    = & - \frac{27}{2} \left[ \frac{1}{4}t^4 - \frac{4}{9}t^3 +
      \frac{1}{6}t^2 \right]_0^1
    \\
    = & - \frac{27}{2} \left( \frac{1}{4} - \frac{4}{9} + \frac{1}{6} \right) 
    \\
    = & - \frac{27}{2} \left( \frac{9}{36} - \frac{16}{36} +
    \frac{6}{36} \right)
    \\
    = & - \frac{27}{2} \left( - \frac{1}{36} \right) 
    \\
    = & \frac{27}{72}
    \\
    = & \frac{3}{8}
    \\
  \end{split}
\end{equation*}

% I(L3) avec le point 1
\underline{Point $x = 1$ :}
\begin{equation*}
  \begin{split}
    \int_0^1 {L_3(t) dt} = & \int_0^1 {\frac{ 9 (t^3 - t^2 + \frac{2}{9} t)}
    {2} dt}
    \\
    = & \frac{9}{2} \int_0^1 { \left(t^3 - t^2 + \frac{2}{9} t \right) dt}
    \\
    = & \frac{9}{2} \left[ \frac{1}{4}t^4 - \frac{1}{3}t^3 +
      \frac{2}{18} t^2 \right]_0^1
    \\
    = & \frac{9}{2} \left(\frac{1}{4} - \frac{1}{3} +
      \frac{2}{18} \right)
    \\
    = & \frac{9}{2} \left( \frac{9}{36} - \frac{12}{36} +
      \frac{4}{36} \right)
    \\
    = & \frac{9}{2} \left( \frac{1}{36} \right)
    \\
    = & \frac{9}{72}
    \\
    = & \frac{1}{8}
    \\
  \end{split}
\end{equation*}

\underline{Résultat :}

\begin{equation*}
    w_0 = \frac{1}{8},
    w_1 = \frac{3}{8},
    w_2 = \frac{3}{8},
    w_3 = \frac{1}{8}
\end{equation*}

% Question 2
\item Ici la méthode ne fonctionne pas. \newline
  Il nous faudrai un point en plus tel que :

  \begin{equation*}
    \int_0^1 {f(t) dt} \approx w_0 f(t_0) + w_1 f(t_1) + w_2 f(t_2) +
    w_3 f(t_3) + w_4 f(t_4)
  \end{equation*}
  
  % Question 3
\item \underline{Formule de quadrature :}
  Faisons un changement de variable. \newline
  On a $x = \frac{t - x_k}{x_{k + 1} - x_k}$ donc $t = (x (x_{k + 1} -
  x_k) + x_k )$
  
  \begin{equation*}
    \begin{split}
      \int_{x_k}^{x_{k+1}} {f(x) dx} & \approx (x_{k + 1} - x_k) \int_0^1
          {f(x (x_{k + 1} - x_k) + x_k ) dx} \\
          & \approx (x_{k + 1} - x_k) \\
          & * \left[ w_0 f( (x_{k +1} - x_k)
            0 + x_k) \right. \\
          & + w_1 f\left( (x_{k +1} - x_k)
            \frac{1}{3} + x_k \right) \\
          & + w_2 f\left( (x_{k +1} - x_k)
            \frac{2}{3} + x_k \right) \\
          & + \left. w_3 f\left( (x_{k +1} - x_k)
            1 + x_k \right) \right]\\
          & \approx (x_{k + 1} - x_k) \\
          & * \left[ \frac{1}{8} f( (x_{k +1} - x_k)
            0 + x_k) \right. \\
          & + \frac{3}{8} f\left( (x_{k +1} - x_k)
            \frac{1}{3} + x_k \right) \\
          & + \frac{3}{8} f\left( (x_{k +1} - x_k)
            \frac{2}{3} + x_k \right) \\
          & + \left. \frac{1}{8} f\left( (x_{k +1} - x_k)
            1 + x_k \right) \right] \\
          & \approx \frac{(x_{k + 1} - x_k)}{8} \left[ f(x_k) + 3
            f\left( \frac{1}{3} x_{k + 1} + \frac{2}{3} x_k \right)
            \right.\\
          & + \left. 3 f\left( \frac{2}{3} x_{k + 1} + \frac{1}{3} x_k
            \right) + f(x_{k + 1}) \right]
    \end{split}
  \end{equation*}


  % Question 4
\item \underline{Formule composite :}

  On a $x_k = a + kh$ et $h = \frac{b - a}{N}$
  
  \begin{equation*}
    \begin{split}
      \int_a^b {f(x) dx} & \approx \sum_{k = 0}^{N - 1}{\int_{x_k}^{x_{k + 1}}{f(x) dx}} \\
      & \approx \sum_{k = 0}^{N - 1} h \frac{(x_{k + 1} - x_k)}{8} \left[ f(x_k) + 3
          f\left( \frac{1}{3} x_{k + 1} + \frac{2}{3} x_k \right)
          \right. \\
      & \left . + 3 f\left( \frac{2}{3} x_{k + 1} + \frac{1}{3} x_k
          \right) + f(x_{k + 1}) \right] \\
      & \approx \sum_{k = 0}^{N - 1} h \frac{(a * (k + 1)h - a - kh)}{8} \left[ f(x_k) + 3
          f\left( \frac{1}{3} x_{k + 1} + \frac{2}{3} x_k \right)
          \right. \\
      & \left . + 3 f\left( \frac{2}{3} x_{k + 1} + \frac{1}{3} x_k
          \right) + f(x_{k + 1}) \right] \\
      & \approx \sum_{k = 0}^{N - 1} \frac{h}{8} \left[ f(x_k) + 3
          f\left( \frac{1}{3} x_{k + 1} + \frac{2}{3} x_k \right)
          \right. \\
      & \left . + 3 f\left( \frac{2}{3} x_{k + 1} + \frac{1}{3} x_k
          \right) + f(x_{k + 1}) \right]
      \end{split}
  \end{equation*}

  % Question 5
\item
  \underline{Nom de la fontion :} \textbf{fourpoints} \newline
  
  \underline{Paramètre d'entrées :}
  \begin{itemize}
  \item \textbf{a} : borne inférieur de l'intervale
  \item \textbf{b} : borne supérieur de l'intervale
  \item \textbf{N} : nombre de pas
  \item \textbf{f} : fonction f
  \end{itemize}

  \underline{Paramètre de sortie :}
  \begin{itemize}
  \item \textbf{res} : résultat de l'intégrale $\int_a^b {f(x) dx}$
  \end{itemize}

  \underline{Explication du code :} \newline
  
  \begin{itemize}
  \item Pour moi $x_k = k * \frac{b - a}{N}$
  \item Le reste est triviale
  \end{itemize}

  \underline{Résultat numérique :} \newline\newline
  \underline{1er test :} \textbf{fourpoints(0, 1, 1, mysquare)}

  \begin{itemize}
  \item $ a = 0 $
  \item $ b = 1 $
  \item $ N = 1 $
  \item f : mysquare fonction $f(x) = x^2$
  \item \textbf{résultat :} $0.3333333$
  \end{itemize}

  \underline{2ème test :} \textbf{fourpoints(0, 10, 1, mysquare)}

  \begin{itemize}
  \item $ a = 0 $
  \item $ b = 10 $
  \item $ N = 1 $
  \item f : mysquare fonction $f(x) = x^2$
  \item \textbf{résultat :} $0.3333333$
  \end{itemize}

  \underline{3ème test :} \textbf{fourpoints(0, 100, 1, mypolynome)}

  \begin{itemize}
  \item $ a = 0 $
  \item $ b = 100 $
  \item $ N = 1 $
  \item f : mypolynome fonction $f(x) = - \frac{1}{3} x^3 +
    \frac{67}{4} x^2 + 7$
  \item \textbf{résultat :} $-2749300.0$
  \end{itemize}

\end{enumerate}

% Exercice 2
\section*{Exercice 2}

\begin{enumerate}

  % Question 1
\item

  \begin{enumerate}

    % Question a
  \item

    \begin{equation*}
      S''_f(x) = 
    \end{equation*}

    % Question b
  \item

    % Question c
  \item

  \end{enumerate}

  % Question 2
\item 
  
\end{enumerate}

% Exercice 3
\section*{Ecercice 3}

\end{document}

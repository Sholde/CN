\documentclass[12pt, letterpaper]{article}
\usepackage[utf8]{inputenc}
\usepackage{amsmath}

\title{TD Calcul Numérique}
\author{BOUTON Nicolas}

\begin{document}

\maketitle

\section{Exercie 1}

\begin{enumerate}
\item Voir la fonction \textbf{pointgauche} dans le fichier "exo1.sci".
\item Voir la fonction \textbf{trapeze} dans le fichier "exo1.sci".
\item Voir la fonction \textbf{int\_simpson} dans le fichier "exo1.sci".
\item Voir la fonction \textbf{sin\_pi\_x} dans le fichier "exo1.sci".
\end{enumerate}

\section{Exercie 2}

On a le système suivant :

$$
\left\{
\begin{array}{l}
  p(-3) = 3 \\
  p(-1) = 7 \\
  p(3) = 7 \\
  p(5) = -3
\end{array}
\right.
$$

Utilisons la métode des \textbf{différence divisé} :

\textit{Premiére étape :}

$$
\left.
\begin{array}{ll}
  x_i & y_i \\
  -3 & 3 \\
  -1 & 7
\end{array}
\right\}
\frac{7 - 3}{-1 - (-3)} = 2
$$

$$
\left.
\begin{array}{ll}
  x_i & y_i \\
  -1 & 7 \\
  3 & 7
\end{array}
\right\}
\frac{7 - 7}{3 - (-1)} = 0
$$

$$
\left.
\begin{array}{ll}
  x_i & y_i \\
  3 & 7 \\
  5 & -3
\end{array}
\right\}
\frac{-3 - 7}{5 - 3} = -5
$$

\textit{Deuxième étape :}

$$
\left.
\begin{array}{ll}
  x_i & y_i \\
  -3 & 2 \\
  3 & 0
\end{array}
\right\}
\frac{0 - 2}{3 -(- 3)} = -\frac{2}{6}
$$

$$
\left.
\begin{array}{ll}
  x_i & y_i \\
  -1 & 0 \\
  5 & -5
\end{array}
\right\}
\frac{-5 - 0}{5 -(- 1)} = -\frac{5}{6}
$$

\textit{Troisième étape :}

$$
\left.
\begin{array}{ll}
  x_i & y_i \\
  -3 & -\frac{2}{6} \\
  5 & -\frac{5}{6}
\end{array}
\right\}
\frac{-\frac{5}{6} - (-\frac{2}{6})}{5 -(- 3)} =
\frac{-\frac{3}{6}}{8} = -\frac{3}{48}
$$

Maintenant on peut calculer $p(x)$ :

\begin{equation*}
\begin{split}
  p(x) = & 3 + 2(x - (-3)) \\
         & + (-\frac{2}{6})(x - (-3))(x - (-1)) \\
         & -\frac{3}{48}(x - (-3))(x - (-1))(x - 3)
\end{split}
\end{equation*}

\begin{equation*}
\begin{split}
  p(x) = & 3 + 2x + 6 \\
         & -\frac{2(x + 3)(x + 1)}{6} \\
         & -\frac{3(x + 3)(x + 1)(x - 3)}{48}
\end{split}
\end{equation*}

\begin{equation*}
\begin{split}
  p(x) = & \frac{144 + 96x + 288}{48} \\
         & -\frac{(2x + 6)(x + 1)}{6} \\
         & -\frac{(3x + 9)(x + 1)(x - 3)}{48}
\end{split}
\end{equation*}

\begin{equation*}
\begin{split}
  p(x) = & \frac{144 + 96x + 288}{48} \\
         & -\frac{2x^2 + 2x + 6x + 6}{6} \\
         & -\frac{(3x^2 + 3x + 9x + 9)(x - 3)}{48}
\end{split}
\end{equation*}

\begin{equation*}
\begin{split}
  p(x) = & \frac{144 + 96x + 288}{48} \\
         & -\frac{2x^2 + 8x + 6}{6} \\
         & -\frac{3x^3 - 9x^2 + 11x^2 - 33x + 9x - 27}{48}
\end{split}
\end{equation*}

\begin{equation*}
\begin{split}
  p(x) = & \frac{144 + 96x + 288}{48} \\
         & -\frac{16x^2 + 64x + 48}{48} \\
         & -\frac{3x^3 + 2x^2 - 24x - 27}{48}
\end{split}
\end{equation*}

\begin{equation*}
\begin{split}
  p(x) = \frac{144 + 96x + 288 - (16x^2 + 64x + 48) - (3x^3 + 2x^2 - 24x - 27)}{48}
\end{split}
\end{equation*}

\begin{equation*}
\begin{split}
  p(x) = \frac{-3x^3 - 18x^2 + 56x + 411}{48}
\end{split}
\end{equation*}

\begin{equation*}
\begin{split}
  p(x) = -\frac{3}{48}x^3 - \frac{6}{16}x^2 + \frac{7}{6}x + \frac{411}{48}
\end{split}
\end{equation*}

\section{Exercie 3}
\section{Exercie 4}
\section{Exercie 5}

\subsection{Euler Explicite}

\begin{enumerate}
\item[a.]

\item[b.]

\end{enumerate}

\subsection{Heun}

\begin{enumerate}
\item[a.]

\item[b.]

\end{enumerate}

\subsection{Euler Implicite}

\begin{enumerate}
\item[a.] Déterminons un polynôme :

$$ y_{i+1} = y_i + h f(y_{i+1})$$
$$ y_{i+1} = y_i + h \frac{1}{2y_{i+1} + 1}$$
$$ y_{i+1} = \frac{y_i (2y_{i+1} + 1) + h}{2y_{i+1} + 1} $$
$$ y_{i+1} (2y_{i+1} + 1) = y_i (2y_{i+1} + 1) + h $$
$$ 2y_{i+1}^2 + y_{i+1} = y_i + 2y_{i+1}y_i + h $$
$$ 2y_{i+1}^2 + y_{i+1} - 2y_{i+1}y_i = y_i + h $$
$$ 2y_{i+1}^2 + y_{i+1} - 2y_{i+1}y_i - y_i - h = 0 $$
$$ 2y_{i+1}^2 + (1 - 2y_i) y_{i+1} - y_i - h = 0 $$

\item[b.] Calculons le descriminant :

$$ \Delta = b^2 - 4ac $$
$$ \Delta = (1 - 2y_i)^2 - [4 * 2 * (- y_i - h)] $$
$$ \Delta = (1 - 2y_i)^2 + 8y_i + 8h $$
$$ \Delta = 1 - 4y_i + (2y_i)^2 + 8y_i + 8h $$
$$ \Delta = 1 + 4y_i + (2y_i)^2 + 8h $$
$$ \Delta = (2y_i + 1)^2 + 8h $$

\item[c.] Déterminons $y_{i+1}$ en fontion de $y_i$ et $h$ :

$$ y_{i+1} = \frac{- b + \sqrt{\Delta}}{2a} $$
$$ y_{i+1} = \frac{2y_i - 1 + \sqrt{(2y_i + 1)^2 + 8h}}{4} $$

\end{enumerate}

\section{Exercie 6}

\section{Exercie 7}

\section{Exercie 8}

\end{document}

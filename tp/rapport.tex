\documentclass[12pt, letterpaper]{article}
\usepackage[utf8]{inputenc}
\usepackage[document]{ragged2e}
\usepackage{amsmath}

\title{TD Calcul Numérique}
\author{BOUTON Nicolas}

\pagenumbering{gobble}

\begin{document}

\maketitle

\section*{Exercice 1}

\begin{enumerate}
\item Voir la fonction \textbf{pointgauche} dans le fichier "exo1.sci".
\item Voir la fonction \textbf{trapeze} dans le fichier "exo1.sci".
\item Voir la fonction \textbf{int\_simpson} dans le fichier "exo1.sci".
\item Voir la fonction \textbf{sin\_pi\_x} dans le fichier "exo1.sci".
\end{enumerate}

\section*{Exercice 2}

On a le système suivant :

$$
\left\{
\begin{array}{l}
  p(-3) = 3 \\
  p(-1) = 7 \\
  p(3) = 7 \\
  p(5) = -3
\end{array}
\right.
$$

Utilisons la métode des \textbf{différence divisé} :

\textit{Premiére étape :}

$$
\left.
\begin{array}{ll}
  x_i & y_i \\
  -3 & 3 \\
  -1 & 7
\end{array}
\right\}
\frac{7 - 3}{-1 - (-3)} = 2
$$

$$
\left.
\begin{array}{ll}
  x_i & y_i \\
  -1 & 7 \\
  3 & 7
\end{array}
\right\}
\frac{7 - 7}{3 - (-1)} = 0
$$

$$
\left.
\begin{array}{ll}
  x_i & y_i \\
  3 & 7 \\
  5 & -3
\end{array}
\right\}
\frac{-3 - 7}{5 - 3} = -5
$$

\textit{Deuxième étape :}

$$
\left.
\begin{array}{ll}
  x_i & y_i \\
  -3 & 2 \\
  3 & 0
\end{array}
\right\}
\frac{0 - 2}{3 -(- 3)} = -\frac{2}{6}
$$

$$
\left.
\begin{array}{ll}
  x_i & y_i \\
  -1 & 0 \\
  5 & -5
\end{array}
\right\}
\frac{-5 - 0}{5 -(- 1)} = -\frac{5}{6}
$$

\textit{Troisième étape :}

$$
\left.
\begin{array}{ll}
  x_i & y_i \\
  -3 & -\frac{2}{6} \\
  5 & -\frac{5}{6}
\end{array}
\right\}
\frac{-\frac{5}{6} - (-\frac{2}{6})}{5 -(- 3)} =
\frac{-\frac{3}{6}}{8} = -\frac{3}{48}
$$

Maintenant on peut calculer $p(x)$ :

\begin{equation*}
\begin{split}
  p(x) = & 3 + 2(x - (-3)) \\
         & + (-\frac{2}{6})(x - (-3))(x - (-1)) \\
         & -\frac{3}{48}(x - (-3))(x - (-1))(x - 3) \\
  p(x) = & 3 + 2x + 6 \\
         & -\frac{2(x + 3)(x + 1)}{6} \\
         & -\frac{3(x + 3)(x + 1)(x - 3)}{48} \\
  p(x) = & \frac{144 + 96x + 288}{48} \\
         & -\frac{(2x + 6)(x + 1)}{6} \\
         & -\frac{(3x + 9)(x + 1)(x - 3)}{48} \\
  p(x) = & \frac{144 + 96x + 288}{48} \\
         & -\frac{2x^2 + 2x + 6x + 6}{6} \\
         & -\frac{(3x^2 + 3x + 9x + 9)(x - 3)}{48} \\
  p(x) = & \frac{144 + 96x + 288}{48} \\
         & -\frac{2x^2 + 8x + 6}{6} \\
         & -\frac{3x^3 - 9x^2 + 11x^2 - 33x + 9x - 27}{48} \\
  p(x) = & \frac{144 + 96x + 288}{48} \\
         & -\frac{16x^2 + 64x + 48}{48} \\
         & -\frac{3x^3 + 2x^2 - 24x - 27}{48} \\
  p(x) = & \frac{144 + 96x + 288 - (16x^2 + 64x + 48) - (3x^3 + 2x^2 -
    24x - 27)}{48} \\
  p(x) = &  \frac{-3x^3 - 18x^2 + 56x + 411}{48} \\
  p(x) = & -\frac{3}{48}x^3 - \frac{6}{16}x^2 + \frac{7}{6}x +
  \frac{411}{48}
\end{split}
\end{equation*}

\section*{Exercice 3}

\begin{enumerate}
\item Calculons $f(-2), f(-1), f(0), f(1), f(2)$ :

  \underline{Calcul de $f(-2)$ :}
  
  \begin{equation*}
    \begin{split}
      f(-2) & = \ln(2\cos \left( \frac{\pi(-2)}{4} \right)^2 + 1) \\
      f(-2) & = \ln(2(0)^2 + 1) \\
      f(-2) & = \ln(1) \\
      f(-2) & = 0
    \end{split}
  \end{equation*}

  \underline{Calcul de $f(-1)$ :}
  
  \begin{equation*}
    \begin{split}
      f(-1) & = \ln(2\cos \left( \frac{\pi(-1)}{4} \right)^2 + 1) \\
      f(-1) & = \ln(2 \left( \frac{\sqrt{2}}{2} \right)^2 + 1) \\
      f(-1) & = \ln(2 \left( \frac{2}{4} \right) + 1) \\
      f(-1) & = \ln(1 + 1) \\
      f(-1) & = \ln(2) \\
      f(-1) & = 0.6931471
    \end{split}
  \end{equation*}

  \underline{Calcul de $f(0)$ :}
  
  \begin{equation*}
    \begin{split}
      f(0) & = \ln(2\cos \left( \frac{\pi(0)}{4} \right)^2 + 1) \\
      f(0) & = \ln(1) \\
      f(0) & = 0
    \end{split}
  \end{equation*}

  \underline{Calcul de $f(1)$ :}
  
  \begin{equation*}
    \begin{split}
      f(1) & = \ln(2\cos \left( \frac{\pi(1)}{4} \right)^2 + 1) \\
      f(1) & = \ln(2 \left( \frac{\sqrt{2}}{2} \right)^2 + 1) \\
      f(1) & = \ln(2 \left( \frac{2}{4} \right) + 1) \\
      f(1) & = \ln(1 + 1) \\
      f(1) & = \ln(2) \\
      f(1) & = 0.6931471
    \end{split}
  \end{equation*}

  \underline{Calcul de $f(2)$ :}
  
  \begin{equation*}
    \begin{split}
      f(2) & = \ln(2\cos \left( \frac{\pi(2)}{4} \right)^2 + 1) \\
      f(2) & = \ln(2(0)^2 + 1) \\
      f(2) & = \ln(1) \\
      f(2) & = 0
    \end{split}
  \end{equation*}

  \underline{Récapitulatif :}

  $$
  \left\{
  \begin{array}{l}
    f(-2) = 0 \\
    f(-1) = ln(2) \\
    f(0) = 0 \\
    f(1) = ln(2) \\
    f(2) = 0
  \end{array}
  \right.
  $$

\item Essayons de trouver un polynôme $p$ de degré inférieur où égale
  à 3 tel que :

  \begin{equation*}
    \left\{
    \begin{array}{l}
      p(-2) = f(-2) \\
      p(-1) = f(-1) \\
      p(-1) = f(1) \\
      p(2) = f(2) 
    \end{array}
    \right.
  \end{equation*}

  \underline{Appliquons la méthodes des différences divisé :}

  \begin{equation*}
    \left\{
    \begin{array}{l}
      p(-2) = 0 \\
      p(-1) = ln(2) \\
      p(-1) = ln(2) \\
      p(2) = 0 
    \end{array}
    \right.
  \end{equation*}

  \textit{Premiére étape :}

  \begin{equation*}
    \left.
    \begin{array}{ll}
      x_i & y_i \\
      -2 & 0 \\
      -1 & ln(2)
    \end{array}
    \right\}
    \frac{ln(2) - O}{-1 - (-2)} = ln(2)
  \end{equation*}

  \begin{equation*}
    \left.
    \begin{array}{ll}
      x_i & y_i \\
      -1 & ln(2) \\
      1 & ln(2)
    \end{array}
    \right\}
    \frac{ln(2) - ln(2)}{1 - (-1)} = 0
  \end{equation*}

  \begin{equation*}
    \left.
    \begin{array}{ll}
      x_i & y_i \\
      1 & ln(2) \\
      2 & 0
    \end{array}
    \right\}
    \frac{0 - ln(2)}{2 - 1} = - ln(2)
  \end{equation*}

  \textit{Deuxième étape :}

  \begin{equation*}
    \left.
    \begin{array}{ll}
      x_i & y_i \\
      -2 & ln(2) \\
      1 & 0
    \end{array}
    \right\}
    \frac{0 - ln(2)}{1 - (-2)} = - \frac{ln(2)}{3}
  \end{equation*}

  \begin{equation*}
    \left.
    \begin{array}{ll}
      x_i & y_i \\
      -1 & 0 \\
      2 & - ln(2)
    \end{array}
    \right\}
    \frac{-ln(2) - 0}{2 - (-1)} = - \frac{ln(2)}{3}
  \end{equation*}

  \textit{Troisième étape :}

  \begin{equation*}
    \left.
    \begin{array}{ll}
      x_i & y_i \\
      -2 & - \frac{ln(2)}{3} \\
      2 & - \frac{ln(2)}{3}
    \end{array}
    \right\}
    \frac{- \frac{ln(2)}{3} - (- \frac{ln(2)}{3})}{2 - (-2)} = 0
  \end{equation*}

  \underline{Essayons de calculer $p$ :}

  \begin{equation*}
    \begin{split}
      p(x) & = 0 \\
      & + ln(2) (x - (-2)) \\
      & - \frac{ln(2)}{3} (x - (-2)) (x - (-1)) \\
      & + 0 (x - (-2)) (x - (-1)) (x - 1) \\
      p(x) & = ln(2) (x + 2) \\
      & - \frac{ln(2)}{3} (x + 2) (x + 1) \\
      p(x) & = ln(2)x + 2ln(2) \\
      & - \frac{ln(2)}{3} (x^2 + x + 2x + 2)\\
      p(x) & = ln(2)x + 2ln(2) \\
      & - \frac{ln(2)x^2 + 3ln(2)x + 2ln(2)}{3}\\
      p(x) & = \frac{3ln(2)x + 6ln(2)}{3} \\
      & - \frac{ln(2)x^2 + 3ln(2)x + 2ln(2)}{3}\\
      p(x) & = \frac{3ln(2)x + 6ln(2) - ln(2)x^2 - 3ln(2)x -
        2ln(2)}{3}\\
      p(x) & = \frac{- ln(2)x^2 + 4ln(2)}{3}\\
      p(x) & = - \frac{ln(2)}{3}x^2 + \frac{4ln(2)}{3}\\
    \end{split}
  \end{equation*}

\item Essayons de trouver un polynôme $q$ de degré inférieur où égale
  à 4 tel que :

  \begin{equation*}
    \left\{
    \begin{array}{l}
      q(-2) = f(-2) \\
      q(-1) = f(-1) \\
      q(0) = f(0) \\
      q(-1) = f(1) \\
      q(2) = f(2) 
    \end{array}
    \right.
  \end{equation*}

  \underline{Appliquons la méthodes des différences divisé :}

  \begin{equation*}
    \left\{
    \begin{array}{l}
      q(-2) = 0 \\
      q(-1) = ln(2) \\
      q(0) = 0 \\
      q(-1) = ln(2) \\
      q(2) = 0 
    \end{array}
    \right.
  \end{equation*}

  \textit{Premiére étape :}

  \begin{equation*}
    \left.
    \begin{array}{ll}
      x_i & y_i \\
      -2 & 0 \\
      -1 & ln(2)
    \end{array}
    \right\}
    \frac{ln(2) - 0}{-1 - (-2)} = ln(2)
  \end{equation*}

  \begin{equation*}
    \left.
    \begin{array}{ll}
      x_i & y_i \\
      -1 & ln(2) \\
      0 & 0
    \end{array}
    \right\}
    \frac{0 - ln(2)}{0 - (-1)} = - ln(2)
  \end{equation*}

  \begin{equation*}
    \left.
    \begin{array}{ll}
      x_i & y_i \\
      0 & 0 \\
      1 & ln(2)
    \end{array}
    \right\}
    \frac{ln(2) - 0}{1 - 0} = ln(2)
  \end{equation*}

  \begin{equation*}
    \left.
    \begin{array}{ll}
      x_i & y_i \\
      1 & ln(2) \\
      2 & 0
    \end{array}
    \right\}
    \frac{0 - ln(2)}{2 - 1} = - ln(2)
  \end{equation*}

  \textit{Deuxième étape :}

  \begin{equation*}
    \left.
    \begin{array}{ll}
      x_i & y_i \\
      -2 & ln(2) \\
      0 & - ln(2)
    \end{array}
    \right\}
    \frac{- ln(2) - ln(2)}{0 - (-2)} = - ln(2)
  \end{equation*}
  
  \begin{equation*}
    \left.
    \begin{array}{ll}
      x_i & y_i \\
      -1 & - ln(2) \\
      1 & ln(2)
    \end{array}
    \right\}
    \frac{ln(2) - (-ln(2))}{1 - (-1)} = ln(2)
  \end{equation*}

  \begin{equation*}
    \left.
    \begin{array}{ll}
      x_i & y_i \\
      0 & ln(2) \\
      2 & - ln(2) 
    \end{array}
    \right\}
    \frac{- ln(2) - ln(2)}{2 - 0} = - ln(2)
  \end{equation*}

  \textit{Troisième étape :}

  \begin{equation*}
    \left.
    \begin{array}{ll}
      x_i & y_i \\
      -2 & - ln(2) \\
      1 & ln(2) 
    \end{array}
    \right\}
    \frac{ln(2) - (-ln(2))}{1 - (-2)} = \frac{2ln(2)}{3}
  \end{equation*}

  \begin{equation*}
    \left.
    \begin{array}{ll}
      x_i & y_i \\
      -1 & ln(2) \\
      2 & - ln(2) 
    \end{array}
    \right\}
    \frac{- ln(2) - ln(2)}{2 - (-1)} = - \frac{2ln(2)}{3}
  \end{equation*}

  \textit{Quatrième étape :}

  \begin{equation*}
    \left.
    \begin{array}{ll}
      x_i & y_i \\
      -2 &  \frac{2ln(2)}{3} \\
      2 & - \frac{2ln(2)}{3}
    \end{array}
    \right\}
    \frac{- \frac{2ln(2)}{3} - \frac{2ln(2)}{3}}{2 - (-2)} = -
    \frac{\frac{4ln(2)}{3}}{4} = - \frac{ln(2)}{3}
  \end{equation*}

  \underline{Essayons de calculer q:}

  \begin{equation*}
    \begin{split}
      q(x) & = 0 \\
      & + ln(2) (x - (-2)) \\
      & - ln(2) (x - (-2)) (x - (-1)) \\
      & + \frac{2ln(2)}{3} (x - (-2)) (x - (-1)) (x - 0) \\
      & - \frac{ln(2)}{3} (x - (-2)) (x - (-1)) (x - 0) (x - 1) \\
      q(x) & = ln(2) (x + 2) \\
      & - ln(2) (x + 2) (x + 1) \\
      & + \frac{2ln(2)}{3} (x + 2) (x + 1) x \\
      & - \frac{ln(2)}{3} (x + 2) (x + 1) x (x - 1) \\
      q(x) & = ln(2) (x + 2) \\
      & - ln(2) (x^2 + x + 2x + 2) \\
      & + \frac{2ln(2)}{3} (x^2 + x + 2x + 2) x \\
      & - \frac{ln(2)}{3} (x^2 + x + 2x + 2) x (x - 1) \\
      q(x) & = ln(2) (x + 2) \\
      & - ln(2) (x^2 + 3x + 2) \\
      & + \frac{2ln(2)}{3} (x^3 + 3x^2 + 2x) \\
      & - \frac{ln(2)}{3} (x^3 + 3x^2 + 2x) (x - 1) \\
      q(x) & = ln(2) (x + 2) \\
      & - ln(2) (x^2 + 3x + 2) \\
      & + \frac{2ln(2)}{3} (x^3 + 3x^2 + 2x) \\
      & - \frac{ln(2)}{3} (x^4 + 3x^3 + 2x^2 - x^3 - 3x^2 - 2x) \\
      q(x) & = ln(2) (x + 2) \\
      & - ln(2) (x^2 + 3x + 2) \\
      & + \frac{2ln(2)}{3} (x^3 + 3x^2 + 2x) \\
      & - \frac{ln(2)}{3} (x^4 + 2x^3 - x^2 - 2x) \\
      q(x) & = ln(2)x + 2ln(2) \\
      & - ln(2)x^2 - 3ln(2)x - 2ln(2) \\
      & + \frac{2ln(2)}{3}x^3 + \frac{6ln(2)}{3}x^2 + \frac{4ln(2)}{3}x \\
      & - \frac{ln(2)}{3}x^4 - \frac{2ln(2)}{3}x^3 + \frac{ln(2)}{3}x^2 + \frac{2ln(2)}{3}x \\
    \end{split}
  \end{equation*}

  \begin{equation*}
    \begin{split}
      q(x) & = - \frac{ln(2)}{3}x^4 \\
      & + \frac{2ln(2)}{3}x^3 - \frac{2ln(2)}{3}x^3 \\
      & - ln(2)x^2 + \frac{6ln(2)}{3}x^2 + \frac{ln(2)}{3}x^2 \\
      & + ln(2)x - 3ln(2)x + \frac{4ln(2)}{3}x + \frac{2ln(2)}{3}x  \\
      & + 2ln(2) - 2ln(2) \\
      q(x) & = - \frac{ln(2)}{3}x^4 \\
      & - \frac{3ln(2) - 6ln(2) - ln(2)}{3}x^2 \\
      & + \frac{3ln(2) - 9ln(2) + 4ln(2) + 2ln(2)}{3}x  \\
      q(x) & = - \frac{ln(2)}{3}x^4 + \frac{4ln(2)}{3}x^2 \\
    \end{split}
  \end{equation*}

\item Différence entre $q$ et $p$ :

  \begin{equation*}
    \begin{split}
      q - p & = - \frac{ln(2)}{3}x^4 + \frac{4ln(2)}{3}x^2 - (- \frac{
        ln(2)}{3}x^2 + \frac{4ln(2)}{3}) \\
      q - p & = - \frac{ln(2)}{3}x^4 + \frac{4ln(2)}{3}x^2 + \frac{
        ln(2)}{3}x^2 - \frac{4ln(2)}{3} \\
      q - p & = - \frac{ln(2)}{3}x^4 + \frac{5ln(2)}{3}x^2 - \frac{4ln(2)}{3} \\
    \end{split}
  \end{equation*}
  
\end{enumerate}

\section*{Exercice 4}

\begin{enumerate}
\item Voir la fonction \textbf{polyLag} dans le fichier "exo4.sci".
\item Voir la fonction \textbf{myinterpol} dans le fichier "exo4.sci".
\end{enumerate}

\section*{Exercice 5}

\begin{enumerate}

\item[\textbf{1.}]\textbf{Euler Explicite}

\begin{enumerate}

\item[a.] Déterminons $y_{i+1}$ en fonction de $y_i$ :

  \begin{equation*}
    \begin{split}
      y_{i+1} & = y_i + h f(y_i) \\
      y_{i+1} & = y_i + h \frac{1}{2y_i + 1} \\
      y_{i+1} & = y_i + \frac{h}{2y_i + 1}
    \end{split}
  \end{equation*}

\item[b.] Voir la fonction \textbf{EulerExplicite} dans le fichier "exo5.sci".
  
\end{enumerate}

\item[\textbf{2.}] \textbf{Heun}

\begin{enumerate}

\item[a.] Déterminons $y_{i+1}$ en fonction de $y_i$ :

  \begin{equation*}
    \begin{split}
      y_{i+1} & = y_i + \frac{h}{2} (f(y_i + h f(y_i)) + f(y_i)) \\
      y_{i+1} & = y_i + \frac{h}{2} \left(\frac{1}{2 \left(y_i + h \frac{1}{2y_i
          + 1}\right) + 1} + \frac{1}{2y_i + 1}\right) \\
      y_{i+1} & = y_i + \frac{h}{2} \left(\frac{1}{2y_i + \frac{2h}{2y_i
          + 1} + 1} + \frac{1}{2y_i + 1}\right)
    \end{split}
  \end{equation*}

\item[b.] Voir la fonction \textbf{Heun} dans le fichier "exo5.sci".

\end{enumerate}

\item[\textbf{3.}] \textbf{Euler Implicite}

\begin{enumerate}
\item[a.] Déterminons un polynôme :

  \begin{equation*}
    \begin{split}
      y_{i+1} & = y_i + h f(y_{i+1}) \\
      y_{i+1} & = y_i + h \frac{1}{2y_{i+1} + 1} \\
      y_{i+1} & = \frac{y_i (2y_{i+1} + 1) + h}{2y_{i+1} + 1} \\
      y_{i+1} (2y_{i+1} + 1) & = y_i (2y_{i+1} + 1) + h \\
      2y_{i+1}^2 + y_{i+1} & = y_i + 2y_{i+1}y_i + h \\
      2y_{i+1}^2 + y_{i+1} - 2y_{i+1}y_i & = y_i + h \\
      2y_{i+1}^2 + y_{i+1} - 2y_{i+1}y_i - y_i - h & = 0 \\
      2y_{i+1}^2 + (1 - 2y_i) y_{i+1} - y_i - h & = 0
    \end{split}
  \end{equation*}

\item[b.] Calculons le descriminant :

    \begin{equation*}
    \begin{split}
      \Delta & = b^2 - 4ac \\
      \Delta & = (1 - 2y_i)^2 - [4 * 2 * (- y_i - h)] \\
      \Delta & = (1 - 2y_i)^2 + 8y_i + 8h \\
      \Delta & = 1 - 4y_i + (2y_i)^2 + 8y_i + 8h \\
      \Delta & = 1 + 4y_i + (2y_i)^2 + 8h \\
      \Delta & = (2y_i + 1)^2 + 8h
    \end{split}
  \end{equation*}

\item[c.] Déterminons $y_{i+1}$ en fontion de $y_i$ et $h$ :

  \begin{equation*}
    \begin{split}
      y_{i+1} & = \frac{- b + \sqrt{\Delta}}{2a} \\
      y_{i+1} & = \frac{2y_i - 1 + \sqrt{(2y_i + 1)^2 + 8h}}{4} 
    \end{split}
  \end{equation*}

\item[d.] Voir la fonction \textbf{EulerImplicite} dans le fichier "exo5.sci".
  
\end{enumerate}

\item[\textbf{4.}] \textbf{Affichage}

\begin{enumerate}
\item[a.] Correspondances des colonnes :

  \begin{itemize}
  \item C2 : Euler Explicite, car les résultats obtenus sont
    inférieurs aux résultats exactes.
  \item C3 : Heun, car c'est la méthode qui à la convergence la plus forte
  \item C4 : Euler Implicite, car les résultats obtenus sont
    supérieurs aux résultats exactes.
  \end{itemize}

\item[b.] Voir la fonction \textbf{AfficheRes} dans le fichier
  "exo5.sci".

\item[c.] Calculons le taux d'erreurs des 3 méthodes :

  \underline{Taux d'erreur avec la méthode d'Euler Explicite :}

  \begin{equation*}
    \begin{split}
      \left\lvert e - e' \right\rvert & = \left\lvert 0,5630146 - 0,5731410
        \right\rvert \\
        & = 0.0101264 \\
        & = 1.01264 \%
    \end{split}
  \end{equation*}

  \underline{Taux d'erreur avec la méthode d'Heun :}

  \begin{equation*}
    \begin{split}
      \left\lvert e - e' \right\rvert & = \left\lvert 0,5630146 - 0,5630245
        \right\rvert \\
        & = 0.0000099 \\
        & = 0.00099 \%
    \end{split}
  \end{equation*}

  \underline{Taux d'erreur avec la méthode d'Euler Implicite :}

  \begin{equation*}
    \begin{split}
      \left\lvert e - e' \right\rvert & = \left\lvert 0,5630146 - 0,5535927
        \right\rvert \\
        & = 0.0094219 \\
        & = 0.94219 \%
    \end{split}
  \end{equation*}

  Nous voyons clairement que les deux méthodes d'Euler sont presque
  équivalentes en termes de convergences exepté qu'elles convergent
  dans des sens différents, les valeurs d'Euler Explicite sont de plus
  en plus grandes que celles attendus et les valeurs d'Euler Implicite
  sont de plus en plus petites des valeurs ettendus.
  On voit également que le taux d'érreur est quasiment nulle pour la
  méthode d'Heun.
\end{enumerate}

\end{enumerate}

\section*{Exercice 6}

\subsection*{Partie I}

Dans cette partie on a le système suivant :

\begin{equation*}
  \left\{
  \begin{array}{l}
    -u''(t) + c(t)u(t) = f(t),  t \in [0,1] \\
    u(0) = \alpha, u'(0) = \beta
  \end{array}
  \right.
\end{equation*}

Et pour tout $t >= 0$ :

\begin{equation*}
  U(t) = \left[
  \begin{array}{l}
    u(t) \\
    u'(t)
  \end{array}
  \right]
\end{equation*}

\begin{enumerate}

\item On a :

  \begin{equation*}
    U'(t) = \left[
    \begin{array}{l}
      u'(t) \\
      u''(t)
    \end{array}
    \right]
  \end{equation*}

  \underline{Calculons $A(t)U(t)+B(t)$ :}

  où \begin{equation*}
    A(t) = \left[
    \begin{array}{ll}
      0 & 1\\
      c(t) & 0
    \end{array}
    \right]
    ,
    B(t) = \left[
    \begin{array}{l}
      0 \\
      -f(t)
    \end{array}
    \right]
  , t \in I.
  \end{equation*}
  
  \begin{equation*}
    A(t)U(t) = \left[
    \begin{array}{ll}
      0 & 1\\
      c(t) & 0
    \end{array}
    \right]
    \left[
    \begin{array}{l}
      u(t) \\
      u'(t)
    \end{array}
    \right]
    =
    \left[
    \begin{array}{l}
      u'(t) \\
      c(t)u(t)
    \end{array}
    \right]
  \end{equation*}

  \begin{equation*}
    A(t)U(t) + B(t) =
    \left[
    \begin{array}{l}
      u'(t) \\
      c(t)u(t)
    \end{array}
    \right]
    +
    \left[
    \begin{array}{l}
      0 \\
      -f(t)
    \end{array}
    \right]
    =
    \left[
    \begin{array}{l}
      u'(t) \\
      c(t)u(t) - f(t)
    \end{array}
    \right]
  \end{equation*}

  or

  \begin{equation*}
    U'(t) = \left[
    \begin{array}{l}
      u'(t) \\
      u''(t)
    \end{array}
    \right]
    =
    \left[
    \begin{array}{l}
      u'(t) \\
      c(t)u(t) - f(t)
    \end{array}
    \right]
  \end{equation*}

  Donc si u est solution de (E) et de (C) alors :

  \begin{equation*}
    U'(t) = A(t)U(t) + B(t)
  \end{equation*}

\item

  \begin{enumerate}
  \item
    On a l'équation suivante :

    \begin{equation*}
      U'(t) = A(t)U(t) + B(t)
    \end{equation*}

    Et on a :
    
    \begin{equation*}
      \begin{split}
        &
        \left\{
        \begin{array}{l}
          U_{i+1} = U_i + h F(t_i, U_i) \\
          U_0 =
          \left[
            \begin{array}{l}
              \alpha \\
              \beta
            \end{array}
            \right]
        \end{array}
        \right.
        \\
        &
        \left\{
        \begin{array}{l}
          U_{i+1} = U_i + h A(t_i) U_i + h B(t_i) \\
          U_0 =
          \left[
            \begin{array}{l}
              \alpha \\
              \beta
            \end{array}
            \right]
        \end{array}
        \right.
        \\
        &
        \left\{
        \begin{array}{l}
          U_{i+1} = U_i + h
          \left[
            \begin{array}{ll}
              0 & 1 \\
              c(t_i) & 0
            \end{array}
            \right]
          U_i + h
          \left[
            \begin{array}{l}
              0 \\
              - f(t_i)
            \end{array}
            \right] \\
          U_0 =
          \left[
            \begin{array}{l}
              \alpha \\
              \beta
            \end{array}
            \right]
        \end{array}
        \right.
      \end{split}
    \end{equation*}

  \item
    On a :

    \begin{equation*}
      \left\{
      \begin{array}{l}
        U_{i+1} = U_i + h A(t_i) U_i + h B(t_i) \\
        U_0 =
        \left[
          \begin{array}{l}
            \alpha \\
            \beta
          \end{array}
          \right]
      \end{array}
      \right.
    \end{equation*}

    Posons
    \begin{equation*}
      U_i =
      \left[
        \begin{array}{l}
          u_i \\
          w_i
        \end{array}
        \right]
    \end{equation*}

    \begin{equation*}
      \begin{split}
      \left[
        \begin{array}{l}
          u_{i+1} \\
          w_{i+1}
        \end{array}
        \right]
      =
      &
       \left[
        \begin{array}{l}
          u_i \\
          w_i
        \end{array}
        \right]
      + h
      \left[
        \begin{array}{ll}
          0 & 1 \\
          c(t_i) & 0
        \end{array}
        \right]
      \left[
        \begin{array}{l}
          u_i \\
          w_i
        \end{array}
        \right]
      + h
      \left[
        \begin{array}{l}
          0 \\
          -f(t_i)
        \end{array}
        \right]
      \\
      =
      &
      \left[
        \begin{array}{l}
          u_i \\
          w_i
        \end{array}
        \right]
      + h
      \left[
        \begin{array}{l}
          w_i \\
          c(t_i) u_i
        \end{array}
        \right]
      + h
      \left[
        \begin{array}{l}
          0 \\
          -f(t_i)
        \end{array}
        \right]
      \end{split}
    \end{equation*}

    On a :

    \begin{equation*}
      \left\{
      \begin{array}{l}
        u_{i+1} = u_i + h w_i \\
        w_{i+1} = w_i + h c(t_i) u_i - h f(t_i)
      \end{array}
      \right.
    \end{equation*}

    \begin{equation*}
      \left\{
      \begin{array}{l}
        w_i = \frac{- u_{i+1} + u_i}{h} \\
        - \frac{u_{i-2} - 2 u_{i-1} - u_i}{h^2} + c(t_i) u_i = f(t_i)
      \end{array}
      \right.
    \end{equation*}

    
  \item
    Voir la fonction \textbf{resouScd} dans "exo6.sci".
    
  \end{enumerate}
  
\end{enumerate}

\subsection*{Partie II}

\begin{enumerate}
\item La formule de taylor à l'ordre 4 donne :

  \begin{equation*}
    \left\{
    \begin{array}{l}
      u(x_{i+1}) = u(x_i) + h_i u'(x_i) + \frac{h_i^2}{2} u''(x_i) +
      \frac{h_i^3}{3!} u^{(3)}(x_i) + \frac{h_i^4}{4} u^{(4)}(x_i) +
      O(h_i^4) \\
      u(x_{i-1}) = u(x_i) - h_{i-1} u'(x_i) + \frac{h_{i-1}^2}{2} u''(x_i) -
      \frac{h_{i-1}^3}{3!} u^{(3)}(x_i) + \frac{h_{i-1}^4}{4} u^{(4)}(x_i) + O(h_i^4)
    \end{array}
    \right.
  \end{equation*}

  $h_i = x_{i+1} - x_i$

  Comme on suppose que la subdivision est uniforme :

  \begin{equation*}
      u(x_{i+1}) - u(x_{i-1}) = 2 u(x_i) + h^2 u''(x_i) + \frac{h^4}{12}
      u^{(4)}(x_i) + O(h^4)
  \end{equation*}

  D'où

  \begin{equation*}
    \begin{split}
      u''(x_i) = & \frac{u(x_{i+1}) - u(x_{i-1}) - 2 u(x_i)}{h^2} +
      O(h^2) \\
      = & \frac{u(x_{i+1}) - u(x_{i-1}) - 2 u(x_i)}{h^2}
    \end{split}
  \end{equation*}

  Cela inspire le problème approché :

  \begin{equation*}
    \left\{
    \begin{array}{l}
      - \frac{u(x_{i+1}) - u(x_{i-1}) - 2 u(x_i)}{h^2} = c(x_i) u_i =
      f(x_i), 1 <= i <= n-1 \\
      u_0 = u_n = 0
    \end{array}
    \right.
  \end{equation*}

  Ce problème approché est un système linéaire :

  \begin{equation*}
    A =
    \left[
    \begin{array}{lllllll}
      \frac{2}{h^2} + c(t_1) & - \frac{1}{h^2} & 0 & 0 & 0 & 0 & 0 \\
      - \frac{1}{h^2} & \frac{2}{h^2} + c(t_2) & - \frac{1}{h^2} & 0 &
      0 & 0 & 0 \\
      0 & - \frac{1}{h^2} & \frac{2}{h^2} + c(t_3) & - \frac{1}{h^2} &
      0 & 0 & 0 \\
      - & - & - & - & - & - & - \\
      0 & 0 & 0 &- \frac{1}{h^2} & \frac{2}{h^2} + c(t_{n-3}) & - \frac{1}{h^2} &
      0 \\
      0 & 0 & 0 & 0 &- \frac{1}{h^2} & \frac{2}{h^2} + c(t_{n-2}) & - \frac{1}{h^2} \\
      0 & 0 & 0 & 0 & 0 &- \frac{1}{h^2} & \frac{2}{h^2} + c(t_{n-1})
    \end{array}
    \right]
  \end{equation*}

  et
  
  \begin{equation*}
    X =
    \left[
      \begin{array}{l}
        u_1 \\
        u_2 \\
        u_3 \\
        - \\
        u_{n-3} \\
        u_{n-2} \\
        u_{n-1}
      \end{array}
      \right]
  \end{equation*}

  et
    \begin{equation*}
      B =
      \left[
        \begin{array}{l}
          f(u_1) \\
          f(u_2) \\
          f(u_3) \\
          - \\
          f(u_{n-3}) \\
          f(u_{n-2}) \\
          f(u_{n-1})
        \end{array}
        \right]
    \end{equation*}

  \item
    \begin{equation*}
      \begin{split}
        h^2 W^T A W = & h^2 W^T \left[
        \begin{split}
          & \frac{2 w_1}{h^2} + c(u_1) w_1 - \frac{w_2}{h^2} \\
          & - \frac{w_1}{h^2} + \frac{2 w_2}{h^2} + c(u_1) w_2 -
          \frac{w_3}{h^2} \\
          & ... \\
          & - \frac{w_{n-3}}{h^2} + \frac{2 w_{n-2}}{h^2} + c(u_{n-2})
          w_{n-2} - \frac{w_{n-1}}{h^2} \\
          & - \frac{w_{n-2}}{h^2} + \frac{2 w_{n-1}}{h^2} + c(u_{n-1}) w_{n-1}
        \end{split}
        \right] \\
        = & h^2 \left(
        \begin{split}
          & w_1 \frac{2 w_1}{h^2} + w_1 c(u_1) w_1 - w_1 \frac{w_2}{h^2} \\
          & - w_2 \frac{w_1}{h^2} + w_2 \frac{2 w_2}{h^2} + w_2 c(u_1) w_2 -
          w_2 \frac{w_3}{h^2} \\
          & ... \\
          & - w_{n-2} \frac{w_{n-3}}{h^2} + w_{n-2} \frac{2
            w_{n-2}}{h^2} + w_{n-2} c(u_{n-2})
          w_{n-2} - w_{n-2} \frac{w_{n-1}}{h^2} \\
          & - w_{n-1} \frac{w_{n-2}}{h^2} + w_{n-1} \frac{2
            w_{n-1}}{h^2} + w_{n-1} c(u_{n-1}) w_{n-1}
        \end{split}
        \right) \\
        = & 
        \begin{split}
          & w_1^2 + h^2 c(t_1) w_1^2 - w_1 w_2 \\
          & - w_1 w_2 + w_2^2 + h^2 c(t_2) w_2^2 -
          w_2 w_3 \\
          & ... \\
          & - w_{n-2} w_{n-3} + w_{n-2}^2 + h^2 c(t_{n-2}) w_{n-2}^2
          - w_{n-2} w_{n-1} \\
          & - w_{n-1} w_{n-2} + w_{n-1}^2 + h^2 c(t_{n-1}) w_{n-1}^2
        \end{split}
        \\
        = & \sum_{i=1}^{n-1} {(2 + h^2c(ih)) w_i^2} - 2 \sum_{i=2}^{n-1} {w_{i-1}w_i}
      \end{split}
    \end{equation*}

    NOTE : $[] = matrixe, () = equation$


  \item
    $A = A^T$ donc A est symétrique car elle est tridiagonale et les
    valeurs sur les diagonales $i = j - 1$ et $i = j + 1$ sont
    constant et égaux.

  \item
    Voir la fonction \textbf{partie2} dans "exo6.sci".

  \item
    Voir la première et dernière valeur du tableau retourner par la fonction.
  
\end{enumerate}

\subsection*{Partie III}

\begin{enumerate}

\item On a :
  \begin{equation*}
    \left\{
    \begin{array}{l}
      -u''(t) + c(t) u(t) = f(t) \\
      u(0) = 0, u(1) = 0
    \end{array}
    \right.
  \end{equation*}
  
  Trouver $u \in V$ tq :

  \begin{equation*}
    \forall v \in V,
    \begin{split}
      & \int_{a}^{b}{u'(x) v'(x) dx} + \int_{a}^{b}{c(x) u(x) v(x) dx} \\
      = & \int_{a}^{b}{f(x) v(x) dx}
    \end{split}
  \end{equation*}

  Le problème s'écrit alors :

  \begin{equation*}
    \left\{
    \begin{array}{l}
      Trouver \quad u \in V \quad tq \\
      \forall v \in V, a(u, v) = l(v)
    \end{array}
    \right.
  \end{equation*}

  où :
  
  \begin{equation*}
    \begin{array}{ll}
      l : & V \rightarrow R \\
      & v \rightarrow \int_{a}^{b}{f(x) v(x) dx}
    \end{array}
  \end{equation*}

  l est linéaire.

  \begin{equation*}
    \begin{array}{ll}
      a : & V x V \rightarrow R \\
      & (w, v) \rightarrow \int_{a}^{b}{w'(x) v'(x) + c(x) w(x) v(x) dx}
    \end{array}
  \end{equation*}

\item Approchons le problème :

  Considérons un espace $V_h \subset V$ de dimension finie.

  Posons $N = dim(V_h)$

  Soit $w_1, ..., w_n$ une base de $V_h$

  Considérons le problème approché :

  \begin{equation*}
    \iff
    \left\{
    \begin{array}{l}
      Trouver \quad u_h \in V_h \quad tq :\\
      \forall v_h \in V_h, a(u_h, v_h) = l(v_h)
    \end{array}
    \right.
  \end{equation*}
  
  \begin{equation*}
    \iff
    \left\{
    \begin{array}{l}
      Trouver \quad u_h \in V_h \quad tq :\\
      \forall 1 \le i \le n, a(u_h, w_i) = l(w_i)
    \end{array}
    \right.
  \end{equation*}

  Cherchons $u_h = \sum_{k=1}^{n}{\alpha_k w_k}$ tq :

  $\iff \forall 1 \le i \le n, a(\sum_{k=1}^{n}{\alpha_k w_k w_i}) = l(w_i)$

  $\iff \forall 1 \le i \le n, \sum_{k=1}^{n}{a(w_k, w_i) \alpha_i} =
  l(w_i)$

  Ce qui nous donne $AU = F$ :
  \begin{equation*}
    \left[
    \begin{array}{lll}
      a(w_1, w_1) & ... & a(w_1, w_n) \\
      ... & ... & ... \\
      a(w_n, w_1) & ... & a(w_n, w_n)
    \end{array}
    \right]
    \left[
      \begin{array}{l}
        \alpha_1 \\
        ... \\
        \alpha_n
      \end{array}
      \right]
    =
    \left[
      \begin{array}{l}
        l(w_1) \\
        ... \\
        l(w_n)
      \end{array}
      \right]
  \end{equation*}

\item
  Soit $w_k = I_h^{-1}(e_k)$ où $\{e_1, ..., e_n\}$ est la base canonique
  de $R^{n-1}$.

  \begin{equation*}
    \begin{split}
      w_k = I_h^{-1}(e_k) & \iff I_h(w_k) = e_k \\
      & \iff
      \begin{split}
        & (w_k(t_1), ..., w_k(t_k), ..., w_k(t_{n-1})) \\
        = & (0, ..., 1, ..., 0)
      \end{split} \\
      & \iff w_k(t_j) = \delta_{k,j} \quad 1 \le j \le n-1
    \end{split}
  \end{equation*}

  Mettre la figure.

  \begin{equation*}
    w_k(t) = 
    \left\{
    \begin{array}{ll}
      \frac{t - t_{k-1}}{t_k - t_{k-1}} & si \quad t \in [t_{k-1},
        t_k] \\
      \frac{t - t_{k+1}}{t_k - t_{k+1}} & si \quad t \in [t_k,
        t_{k+1}] \\
      0 & si \quad t \notin [t_{k-1}, t_{k+1}]
    \end{array}
    \right.
  \end{equation*}

  Si la subdivision est uniforme on a :

  \begin{equation*}
    w_k(t) = 
    \left\{
    \begin{array}{ll}
      \frac{t - t_{k-1}}{h} & si \quad t \in [t_{k-1},
        t_k] \\
      - \frac{t - t_{k+1}}{h} & si \quad t \in [t_k,
        t_{k+1}] \\
      0 & sinon
    \end{array}
    \right.
  \end{equation*}

\item

\item
  Lorque $|i - j| > 1$, $a(i, j) = 0$. A est une matrice tridiagonale.

\item

\item
  

\end{enumerate}

\section*{Exercice 7}

\section*{Exercice 8}

\end{document}

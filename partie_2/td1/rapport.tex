\documentclass{article}

\title{TP Calcul Numérique}
\author{Nicolas BOUTON}

\begin{document}

\maketitle

\section*{Exercice 2}

Nous remarquons 3 chose :

\begin{itemize}
\item Augmentation du conditionnement avec la taille
\item Il faut considéré l'erreur arrière entaché du conditionnement
  pour avoir une approximation des résultats
\item Nous manipulons des algorithmes avec certaines complexité qui
  peuvent être algorithmique ou bien de mémoire
\end{itemize}

Le code source se trouve dans \textbf{exo2.sci}.

\section*{Exercice 3}

Le produit \textbf{matrice x matrice} à une complexité cubique.
Nous remarquons dans un premier temps que l'appel au fonction de
\textbf{Scilab} pour des calculs \textbf{vecteur x vecteur} ou bien
\textbf{vecteur x matrice} au lieu de déroulé nous même les bloucles
est beaucoup plus performant. Il est d'autant plus performant si la
taille des matrices est grandes. \textbf{Scilab} appelle des noyaux de
calcul optimizé tel que blass. \newline

\underline{Graphiques :} à faire \newline

Normalement les graphique sont fait, il suffit que je les
sauvegarde. On voit que \textbf{matmat3b} est \textbf{cubique},
\textbf{matmat2b} est \textbf{quadratique} et \textbf{matmat1b} est
\textbf{linéaire}.

\begin{itemize}
\item Le code des fonction se trouve dans \textbf{opmat.sci}.
\item Le code de test se trouve dans \textbf{exo3.sci}.
\end{itemize}

\section*{Exercice 4}

Bah ça marche ...

\section*{Annexe}

Dépot github : https://github.com/Sholde/CN/partie\_2

\end{document}

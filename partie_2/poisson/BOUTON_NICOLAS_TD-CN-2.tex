% Created 2020-12-03 jeu. 20:36
% Intended LaTeX compiler: pdflatex
\documentclass[11pt]{article}
\usepackage[utf8]{inputenc}
\usepackage[T1]{fontenc}
\usepackage{graphicx}
\usepackage{grffile}
\usepackage{longtable}
\usepackage{wrapfig}
\usepackage{rotating}
\usepackage[normalem]{ulem}
\usepackage{amsmath}
\usepackage{textcomp}
\usepackage{amssymb}
\usepackage{capt-of}
\usepackage{hyperref}
\author{Nicolas BOUTON}
\date{2020}
\title{TP de Calcul Numérique}
\hypersetup{
 pdfauthor={Nicolas BOUTON},
 pdftitle={TP de Calcul Numérique},
 pdfkeywords={},
 pdfsubject={},
 pdfcreator={Emacs 27.1 (Org mode 9.3)}, 
 pdflang={English}}
\begin{document}

\maketitle
\tableofcontents


\section{Exercice 1}
\label{sec:orgc2bf58d}

Developpement limité : 

\begin{equation*}
\begin{split}
T(x_i + h) & = T(x_i) + h \left(\frac{\delta T}{\delta x} \right)_i + h^2 \left(\frac{\delta^2 T}{\delta x^2} \right)_i + O(h^2) \\
T(x_i - h) & = T(x_i) - h \left(\frac{\delta T}{\delta x} \right)_i + h^2 \left(\frac{\delta^2 T}{\delta x^2} \right)_i + O(h^2)
\end{split}
\end{equation*}

On somme et on inverse le signe :

\begin{equation*}
\begin{split}
- T(x_i + h) + 2 T(x_i) - T(x_i - h) = - h^2 \left(\frac{\delta^2 T}{\delta x^2} \right)_i + O(h^2) \\
\frac{- T(x_i + h) + 2 T(x_i) - T(x_i - h)}{h^2} = - \left(\frac{\delta^2 T}{\delta x^2} \right)_i
\end{split}
\end{equation*}

Or on a :

\begin{equation*}
- k \left( \frac{\delta^2 T}{\delta x^2} \right)_i = g_i, k > 0
\end{equation*}

On se permet de négligé k car c'est une constante dans nos prochain calcul :

\begin{equation*}
\begin{split}
- T(x_i + h) + 2 T(x_i) - T(x_i - h) = h^2 g_i
\end{split}
\end{equation*}

On écrit le système d'équation : 

\begin{equation*}
\begin{array}{ll}
u_0 = T_0 & i = 0 \\
- u_0 + 2 u_1 - u_2 = h^2 g_1 & i = 1\\
... & ... \\
- u_{k-1} + 2 u_k - u_{k+1} = h^2 g_k & i = k\\
... & ... \\
- u_{n-1} + 2 u_n - u_{n+1} = h^2 g_n & i = n\\
u_n = T_n & i = n + 1 \\
\end{array}
\end{equation*}

Avec les conditions aux bords on obtient :

\begin{equation*}
2 u_1 - u_2 = h^2 g_1 + T_0 \\
- u_{n-1} + 2 u_n = h^2 g_n + T_n \\
\end{equation*}

Donc on explicite le système linéaire \(Au = g\) :

\begin{equation*}
A = \left[
\begin{array}{ccccccc}
2 & -1 & 0 & - & - & - & 0 \\
-1 & 2 & -1 & . &  &  & |  \\
0 & -1 & . & . & . &  & |  \\
| & . & . & . & . & . & |  \\
| & & . & . & . & -1 & 0  \\
| & & & . & -1 & 2 & -1  \\
0 & - & - & - & 0 & -1 & 2 \\
\end{array}
\right]
\end{equation*}

\begin{equation*}
u = \left[
\begin{array}{c}
T_1 \\
| \\
T_n \\
\end{array}
\right]
\end{equation*}

\begin{equation*}
g = \left[
\begin{array}{c}
h^2 T_1 + T_0 \\
h^2 T_2 \\
| \\
h^2 T_{n-1} \\
h^2T_n + T_1\\
\end{array}
\right]
\end{equation*}

Comme il n'y a pas de source de chaleur, on a \(\forall\) i \(\in\) [ 1, n
] : h\textsuperscript{2} g\textsubscript{i} = 0

D'où g = \left[
\begin{array}{c}
T_0 \\
0 \\
| \\
0 \\
T_1 \\
\end{array}
\right]

Et la solution qnalytique qui se déguage est : 

$$ T(x) = T_0 + x (T_1 - T_0) $$

\section{Exercice 2}
\label{sec:org50083dc}
\subsection{Arch}
\label{sec:orgec442c3}
\subsubsection{Bibliothèque}
\label{sec:org7254db5}

Pour l'intallation des bibliothèque \textbf{cblas} et \textbf{lapacke} :
\begin{verbatim}
$ sudo pacman -S cblas lapacke
\end{verbatim}

\subsubsection{Makefile}
\label{sec:org71efc17}

Il faut modifier la ligne qui link les librairie en linkant la
bibliothèque \textbf{cblas}:

\begin{verbatim}
#
# -- librairies
LIBS=-llapacke -lcblas -lm
\end{verbatim}

\section{Exercice 3}
\label{sec:org2a3f00e}
\subsection{Question 1}
\label{sec:org466d901}

Les matrices pour utilisé \textbf{BLAS} et \textbf{LAPACK} en \textbf{C} se font de la
même manière que les tableaux en \textbf{C}. C'est-à-dire que que pour une
matrice donné, il faut la stocké en \textbf{1 dimmension}.

Par exemple en C :
\begin{verbatim}
double[2][2] = { {1, 2}, {3, 4} };
\end{verbatim}

Pour \textbf{BLAS} et \textbf{LAPACK} :

\begin{verbatim}
double[4] = { 1, 2, 3, 4 };
\end{verbatim}

\subsection{Question 2}
\label{sec:org1adee99}

\begin{itemize}
\item Les constantes \(LAPACK\_ROW\_MAJOR\) et \(LAPACK\_COL\_MAJOR\)
signifie la priorité ligne ou colonne respectivement de la
représentation de la matrice.
\item Effectivement, comme il faut utilisé des tableaux en \textbf{1 dimension}
il faut préciser si on a utilisé une priorité ligne ou colonne pour
ranger la matrice.
\end{itemize}

\subsection{Question 3}
\label{sec:org107d882}

\begin{itemize}
\item De ce que j'ai compris, c'est un argument qui permet de savoir si
dans la représentatin de la matrice, les éléments des lignes ou des
colonnes, suivant la priorité choisis, sont contigue.

\item C'est-à-dire qu'il dois y avoir le même nombre d'élément pour
chaque colonne ainsi que dans chaque ligne.
\end{itemize}

\subsection{Question 4}
\label{sec:orgcd666c8}
\subsubsection{Résumé}
\label{sec:org462cb27}

La fonction \(LAPACKE\_dgbsv\) permet de calculer le résultat d'un
système linéaire du type \(A * X = B\), avec \textbf{X} l'inconnu, \textbf{A} une
matrice et \textbf{B} le second membre, où \textbf{X} et \textbf{B} peuvent être des
vecteurs ou des matrices.

\subsubsection{Argument}
\label{sec:org010939b}

Elle prend en argument la dimension de la matrice, le nombre du
sous-diagonnale ainsi que de sur-diagonnale.

\subsubsection{Implémentation}
\label{sec:orgca7c1c3}

Cette fonction implémente une décomposition \textbf{LU} à pivot partiel et
la méthode de dessente et de remonté.

\subsubsection{Sources}
\label{sec:org86015d0}

\url{http://www.math.utah.edu/software/lapack/lapack-d/dgbsv.html}

\subsection{Question 5}
\label{sec:orgdb19e77}
\section{Annexe}
\label{sec:org4a46c2c}

Dépôt : \url{https://github.com/Sholde/CN/tree/master/partie\_2/poisson}
\end{document}
